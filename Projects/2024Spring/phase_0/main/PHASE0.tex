\documentclass[]{article}
\usepackage{graphicx}
\usepackage[svgnames]{xcolor} 
\usepackage{fancyhdr}

\usepackage[hidelinks]{hyperref}
\usepackage{enumitem}
\usepackage[many]{tcolorbox}
\usepackage{listings }
\usepackage[a4paper, total={6in, 8in}]{geometry}
\usepackage{afterpage}
\usepackage{amssymb}
\usepackage{pdflscape}
\usepackage{textcomp}
\usepackage{xecolor}
\usepackage{rotating}
\usepackage[Kashida]{xepersian}
\usepackage[T1]{fontenc}
\usepackage{tikz}
\usepackage[utf8]{inputenc}
\usepackage{PTSerif} 
\usepackage{seqsplit}
\usepackage{changepage}



\usepackage{listings}
\usepackage{xcolor}
\usepackage{sectsty}
 
\definecolor{codegreen}{rgb}{0,0.6,0}
\definecolor{codegray}{rgb}{0.5,0.5,0.5}
\definecolor{codepurple}{rgb}{0.58,0,0.82}
\definecolor{backcolour}{rgb}{0.95,0.95,0.92}
 
\NewDocumentCommand{\codeword}{v}{
\texttt{\textcolor{blue}{#1}}
}
\lstset{language=java,keywordstyle={\bfseries \color{blue}}}

\lstdefinestyle{mystyle}{
    backgroundcolor=\color{backcolour},   
    commentstyle=\color{codegreen},
    keywordstyle=\color{magenta},
    numberstyle=\tiny\color{codegray},
    stringstyle=\color{codepurple},
    basicstyle=\ttfamily\normalsize,
    breakatwhitespace=false,         
    breaklines=true,                 
    captionpos=b,                    
    keepspaces=true,                 
    numbers=left,                    
    numbersep=5pt,                  
    showspaces=false,                
    showstringspaces=false,
    showtabs=false,                  
    tabsize=2
}

\lstset{style=mystyle}

 \settextfont[BoldFont={XB Zar bold.ttf}]{XB Zar.ttf}


\setlatintextfont[Scale=1.0,
 BoldFont={LiberationSerif-Bold.ttf}, 
 ItalicFont={LiberationSerif-Italic.ttf}]{LiberationSerif-Regular.ttf}





\newcommand{\inputsample}[1]{
    ~\\
    \textbf{ورودی نمونه}
    ~\\
    \begin{tcolorbox}[breakable,boxrule=0pt]
        \begin{latin}
            \large{
                #1
            }
        \end{latin}
    \end{tcolorbox}
}

\newcommand{\outputsample}[1]{
    ~\\
    \textbf{خروجی نمونه}

    \begin{tcolorbox}[breakable,boxrule=0pt]
        \begin{latin}
            \large{
                #1
            }
        \end{latin}
    \end{tcolorbox}
}

\newenvironment{changemargin}[2]{%
\begin{list}{}{%
\setlength{\topsep}{0pt}%
\setlength{\leftmargin}{#1}%
\setlength{\rightmargin}{#2}%
\setlength{\listparindent}{\parindent}%
\setlength{\itemindent}{\parindent}%
\setlength{\parsep}{\parskip}%
}%
\item[]}{\end{list}}


\definecolor{foldercolor}{RGB}{124,166,198}
\definecolor{sectionColor}{HTML}{ff5e0e}
\definecolor{subsectionColor}{HTML}{008575}


\defpersianfont\titr[Scale=1.5]{Lalezar-Regular.ttf}

\sectionfont{\color{sectionColor}}  % sets colour of sections



\subsectionfont{\color{subsectionColor}}  % sets colour of sections

\renewcommand{\labelitemii}{$\star$}
\renewcommand{\labelitemiii}{$\star\#$}

\renewcommand{\baselinestretch}{1.1}
\setlength{\parskip}{1.2pt}

\begin{document}


%%% title pages
\begin{titlepage}
\begin{center}

\textbf{ \Huge{به نام خدا} }
        
\vspace{0.2cm}

\includegraphics[width=0.4\textwidth]{sharif1.png}\\
\vspace{0.2cm}
\textbf{ \Huge{\emph درس برنامه‌سازی پیشرفته} }\\
\vspace{0.25cm}
\textbf{ \Large{ فاز صفر پروژه} }
\vspace{0.2cm}
       
 
      \large \textbf{دانشکده مهندسی کامپیوتر}\\\vspace{0.1cm}
    \large   دانشگاه صنعتی شریف\\\vspace{0.2cm}
       \large   ﻧﯿﻢ سال دوم 99-98 \\\vspace{0.10cm}
      \noindent\rule[1ex]{\linewidth}{1pt}
اساتید:\\
    \textbf{{مهدی مصطفی‌زاده، ایمان عیسی‌زاده، امیر ملک‌زاده، علی چکاه}}



    \vspace{0.20cm}

   مهلت ارسال:\\
    \textbf{{28 فروردین - }}
    \textbf{{ساعت 23:59:59}}

    \vspace{0.10cm}
نگارش:\\
    \textbf{{احمد سلیمی و امیرمهدی نامجو}}
    

\end{center}
\end{titlepage}
%%% title pages


%%% header of pages
\newpage
\pagestyle{fancy}
\fancyhf{}
\fancyfoot{}
\cfoot{\thepage}
\lhead{فاز صفر}
\rhead{\includegraphics[width=0.1\textwidth]{sharif.png}\\
دانشکده مهندسی کامپیوتر
}
\chead{پروژه برنامه‌سازی پیشرفته}
%%% header of pages
\renewcommand{\headrulewidth}{2pt}

\KashidaOff


 \Large \textbf{\\\\
}






\section* {{\titr مقدمه - فاز صفر یعنی چی؟}}

در این فاز شما باید مقدمات پروژه را حاضر کنید. این مقدمات شامل ابزارهای مورد استفاده در پروژه و همچنین طراحی معماری پروژه توسط دیاگرام \lr{UML} است. مراحل این فاز عبارت اند از:

\begin{enumerate}
\item
\hyperref[subsec:trello]{\textcolor{blue}{راه‌اندازی \lr{Trello}}}

\item
\hyperref[subsec:github]{\textcolor{blue}{راه اندازی مخزن \lr{GitHub}}}

\item
\hyperref[subsec:uml]{\textcolor{blue}{طراحی \lr{UML} برای منطق پروژه}}


\end{enumerate}

در بخش بعد، هر یک از این موارد شرح داده شده اند.

\newpage
\section*{{\titr کارهایی که باید در فاز صفر انجام دهید}}


\subsection*{{\titr راه‌اندازی Trello}}

\label{subsec:trello}

مستند آموزشی این بخش، در پیوست این فاز موجود است.

در این پروژه، شما بایستی کارهای هر شخص را در  \lr{Trello} مشخص کرده و پس از انجام آن کار توسط هر شخص، تسک مربوط به آن را  به بخش تسک‌های انجام شده منتقل کنید.

هر یک از اعضای تیم پروژه، باید در سایت
\href{https://trello.com/}{\textcolor{blue}{\lr{trello.com}}}
  یک حساب کاربری بسازد. سپس یکی از اعضای تیم، یک \lr{board} جدید ساخته و مابقی اعضا را به آن دعوت کند.

حال در \lr{board} ساخته شده، تسک‌های زیر را معین کرده، و هر کدام را به یک نفر اختصاص دهید:

\begin{itemize}

\item
ایجاد مخزن \lr{Github} و \lr{commit} نام و شماره دانشجویی در \lr{README.md} (تسک به نفر اول \lr{assign} شود)

\item
کامیت نام و شماره دانشجویی در \lr{README.md} (تسک به نفر دوم \lr{assign} شود)

\item
کامیت نام و شماره دانشجویی در \lr{README.md} (تسک به نفر سوم \lr{assign} شود)

\end{itemize}

پس از انجام تسک ها آن‌ها را به بخش \lr{done} منتقل کنید.

در ادامه‌ی کار نیز باید از این ابزار استفاده کنید ‌و تسک‌های خود را مدیریت کنید و در تحویل حضوری هر فاز، بورد شما بررسی می‌شود.


\newpage
\subsection*{{\titr راه‌اندازی مخزن GitHub}}
\label{subsec:github}

همانطور که می‌دانید برای پروژه لازم است با گروهتان بر روی یک مخزن \lr{(repository)} گیت فعالیت کنید. برای ساختن این مخزن، کافیست وارد
 \href{https://classroom.github.com/g/5Uo3x-M4}{\textcolor{blue}{\underline{این لینک}}} 
 شوید.

ابتدا با لیستی مواجه می‌شوید که شماره دانشجویی تمام افراد در آن موجود است. شماره دانشجویی خود را بیابید و بر روی آن کلیک کنید.

در صفحه‌ی بعد شما باید تیم خود را انتخاب کنید. چنانچه نفر اول گروه خود (سازنده‌ی مخزن) هستید، باید یک تیم بسازید. تنها شماره‌ی گروه پروژه خود را در قسمت نام تیم وارد کنید و تیم را بسازید. نفرات بعدی گروه شما، باید تیم‌شان را از لیست تیم‌های موجود انتخاب کنند و نیازی به ایجاد تیم ندارند.

پس از این مراحل مخزن شما آماده خواهد شد و لینک آن در اختیارتان قرار خواهد گرفت. پس از آماده شدن این مخزن، هر یک از اعضای پروژه باید نام و شماره دانشجویی خود را به فایل \lr{README.md} اضافه کند.

\newpage
\subsection*{{\titr طراحی UML برای منطق پروژه}}
\label{subsec:uml}

مستند فاز اول پروژه به زودی منتشر خواهد شد.

برای این بخش، شما لازم است مستند فاز اول پروژه را به طور کامل و دقیق مطالعه نمایید. سپس یک \lr{UML} مناسب برای تمام بخش‌های منطق پروژه طراحی کنید. فایل ارسالی شما باید یک فایل \lr{zip}، حاوی تصاویر مربوط به \lr{UML} باشد.

نکته: برای رسم uml می توانید از هر ابزار دلخواهی استفاده کنید.

مستند آموزشی \lr{UML} نیز در پیوست این فاز موجود است.



\end{document}









